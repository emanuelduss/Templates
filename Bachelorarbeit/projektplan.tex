\chapter{Projektmanagementplan}

\section{Einführung}

\subsubsection*{Zweck dieses Dokuments}

\subsubsection*{Zweck und Ziel dieser Arbeit}

\subsubsection*{Lieferumfang}

\subsubsection*{Annahmen und Einschränkungen}

\section{Projektorganisation}

\begin{tabular}[t]{|l|l|l|} \hline
\textbf{Name} & \textbf{E-Mail} & \textbf{Aufgabe und Verantwortungen} \\ \hline \hline
Kollege Essig         & kessig@hsr.ch    & Informationsbeschaffung, Entwicklung, Dokumentation \\ \hline
Emanuel Duss          & eduss@hsr.ch     & Informationsbeschaffung, Entwicklung, Dokumentation \\ \hline
\end{tabular}

\subsubsection*{Externe Schnittstellen}

\begin{tabular}[t]{|l|l|l|} \hline
\textbf{Name} & \textbf{E-Mail} & \textbf{Aufgabe und Verantwortungen} \\ \hline \hline
Rainer Zufall  & rzufall@hsr.ch  & Betreuer \\ \hline
\end{tabular}

\subsubsection*{Sitzungen}

\begin{itemize}
  \item 23. Mai 2015; 23:00-23:30
\end{itemize}

\section{Managementabläufe}

\subsection{Zeiterfassung}

\subsection{Allgemeine Bemerkungen zur Zeitplanung}

\subsection{Arbeitspakete und zeitliche Planung}

% \begin{landscape}
% \begin{footnotesize}
%
% \begin{figure}[h]
%   \centering
%   \includegraphics[width=24cm]{images/zeitplanung.png}
%   \caption{Uebersicht über die geplanten Arbeitspakete}
%   \label{fig:ArbeitspaketeGeplant}
% \end{figure}
% 
% \begin{figure}[h]
%   \centering
%   \includegraphics[width=24cm]{images/gantt.png}
%   \caption{Uebersicht über die geplanten Arbeitspakete im GANTT Format}
%   \label{fig:ArbeitspaketeGeplantGANTT}
% \end{figure}
%
% \end{footnotesize}
% \end{landscape}
% \newpage

\subsection{Meilensteine}

\begin{tabular}[t]{|l|l|p{6.5cm}|l|}\hline
\textbf{Meilenstein} & \textbf{Datum} & \textbf{Ziele} & \textbf{Dauer} \\ \hline \hline
M0 Projektstart & 16.02.2015 & & - \\ \hline
M1 Inception & 08.03.2015 &
  \begin{itemize}
    \item Projektplan erstellen
    \item Arbeitsumgebung einrichten
  \end{itemize} & 1 Woche \\ \hline
M2 Elaboration 1 & 05.04.2015 &
  \begin{itemize}
    \item Foo
    \item Foo
    \item Foo
  \end{itemize} & 6 Wochen \\ \hline
M3 Elaboration 2 & 19.04.2015 &
  \begin{itemize}
    \item Foo
    \item Foo
    \item Foo
  \end{itemize} & 2 Wochen \\ \hline
M4 Construction 1 & 31.05.2015 &
  \begin{itemize}
    \item Foo
    \item Foo
    \item Foo
  \end{itemize} & 6 Wochen \\ \hline
M5 Construction 2 & 03.06.2015 &
  \begin{itemize}
    \item Foo
    \item Foo
    \item Foo
  \end{itemize} & 30 Stunden \\ \hline
M6 Transition & 09.06.2015 &
  \begin{itemize}
    \item Foo
    \item Foo
    \item Foo
  \end{itemize} & 60 Stunden \\ \hline
M7 Projektende & 12.06.2015 & & - \\ \hline

\end{tabular}

\captionof{table}{Meilensteine und deren Ziele}
\label{tab:MeilensteineZiele}

\newpage
\section{Risikomanagement}

\begin{tabular}[t]{|p{3cm}|p{3cm}|r|r|r|p{3cm}|p{3cm}|}\hline
\textbf{Risiko} &
  \textbf{Auswirkung} &
  \begin{sideways} \textbf{Wahrscheinlichkeit } \end{sideways} &
  \begin{sideways}\textbf{Schaden} \end{sideways} &
  \begin{sideways}\textbf{Risiko} \end{sideways} &
  \textbf{Vorbeugung} & \textbf{Konsequenzen} \\ \hline \hline
Datenverlust &
  verlorene Arbeit &
  0.1 & 0.9 & 0.1 &
  regelmässige Backups &
  Arbeit in Sonderschicht nachholen \\ \hline
Ausfall eines Projektmitarbeiters &
  Nichteinhaltung des Terminplans &
  0.1 & 0.9 & 0.1 &
  Nicht vermeidbar &
  Anpassung der Projektziele \\ \hline
\end{tabular}
\captionof{table}{Risiken}
\label{tab:Risiken}

Sollte trotz den vorbeugenden Massnahmen ein zeitlicher Schaden entstehen, muss
die Projektplanung unter Umständen angepasst werden.

\section{Qualitätsmanagement}

\section{Projektstandverfolgung}

\subsection{Meilenstein 1: Inception}

\section{Zeitauswertung}

\subsection{Projektstunden pro Woche}

\subsection{Projektstunden aufsummiert}

\subsection{Projektstunden pro Projektmitglied}

\subsection{Stunden pro Tätigkeitsbereich}
